% vim:ft=tex:
%
\documentclass[8pt]{beamer}

\usepackage{times}
\usepackage{graphicx}
\usepackage[]{geometry}
\usepackage{booktabs}
\usepackage{hyperref}

\usepackage{tikz}

\usetheme{Montpellier}
\usecolortheme{beaver}

\usepackage{xcolor}

\definecolor{asparagus}{rgb}{0.53, 0.66, 0.42}
\definecolor{aurometalsaurus}{rgb}{0.43, 0.5, 0.5}
\definecolor{beige}{rgb}{0.96, 0.96, 0.86}
\setbeamercolor{titlelike}{parent=palette primary,fg=asparagus}
\setbeamercolor*{palette primary}{bg=beige, fg=asparagus}
\setbeamercolor*{palette secondary}{bg=beige!90!aurometalsaurus, fg=asparagus}
\setbeamercolor*{palette tertiary}{bg=aurometalsaurus, fg=asparagus}
\setbeamercolor*{palette quaternary}{bg=beige, fg=asparagus}
\setbeamercolor{frametitle}{bg=aurometalsaurus!10!beige}
\setbeamercolor{frametitle right}{bg=aurometalsaurus!60!beige}
\setbeamertemplate{itemize item}{\color{asparagus}$\blacktriangleright$}

\hypersetup{colorlinks,urlcolor=asparagus}

\usepackage{pdfpages}

\usepackage{enumitem}

\setlist[itemize]{itemsep=10pt, label={\color{asparagus}$\blacktriangleright$}}

\setbeamertemplate{navigation symbols}{}

\title{DNA analysis in population genetics\\
Practical 2, the coalescent}

\date{Februrary, 25, 2020}

\author{
    \texorpdfstring{
        \begin{columns}
              \column{.45\linewidth}
              \centering
              Chedly Kastally\\
              \href{mailto:chedly.kastally@oulu.fi}{chedly.kastally@oulu.fi}
              \column{.45\linewidth}
              \centering
              Tanja Pyh{\"a}j{\"a}rvi\\
              \href{mailto:tanja.pyhajarvi@oulu.fi}{tanja.pyhajarvi@oulu.fi}
        \end{columns}
    % \texorpdfstring{}
    }
    {Kastally, Chedly \and Tanja Pyh{\"a}j{\"a}rvi}
}


\begin{document}

\begin{frame}

\maketitle

\end{frame}

{
\setbeamercolor{background canvas}{bg=}
\includepdf[width=1.5\textwidth]{demographic_models.pdf}
}

\begin{frame}
    \frametitle{Demographic processes}

    \large

    \begin{itemize}
    
        \item<1-> In population genetics, we are interested in understanding how
            various processes, such as demographic processes, impact genetic
            variation.

        \item<2-> To this end, we use models, such as the \textit{coalescent}

    \end{itemize}

\end{frame}


{
\setbeamercolor{background canvas}{bg=}
\includepdf[width=1\textwidth]{genealogy.pdf}
}

\begin{frame}
    \frametitle{The coalescent}

    \large

    \begin{itemize}
    
        \item<1-> The coalescent is a model of how \textit{genetic elements} \textit{sampled from a population} originated from a common ancestor

        \item<2-> In a coalescent model, we go backward in time, from
            the present to the time of the most recent ancestor (TMRCA).

    \end{itemize}

\end{frame}

\begin{frame}
    \frametitle{Concepts}

    \large

    Some additional concepts

    \vspace{1em}

    \begin{itemize}
    
        \item<1-> Summary Statistics: $\pi$, $\theta$...

        \item<2-> Tajima's D

        \item<3-> The Site Frequency Spectrum

    \end{itemize}

\end{frame}

\begin{frame}

    \frametitle{Today}
    \large

    \begin{itemize}
    
        \item<1-> You will explore different aspects of the coalescent
            using a package implemented in R:\ \textit{coala}

        \item<2-> You will use this tool to model different and simple
            coalescent models of population

        \item<3-> You will simulate coalescent processes, and distribution of
            statistics using these models, and explore how variation in
            those models affect those statistics and your simulations.

    \end{itemize}

\end{frame}

\begin{frame}
    \frametitle{Instructions}

    \begin{itemize}
    
    \item<1-> Instructions in Moodle: 2\_Coalescent\_Practical.pdf;
    explore as much as possible
    \vspace{1em}

    \item<2-> Questions to be answered in Moodle
    \vspace{1em}

    \item<3-> An R script is also expected and has to be submitted
    in Moodle

    \end{itemize}

\end{frame}



\end{document}
